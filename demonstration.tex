\documentclass[8pt, handout, aspectratio = 1510]{beamer}

\usetheme{DarkModern}
\setbeameroption{hide notes}

\title{The \textit{DarkModern} Theme for Beamer}
\subtitle{Showcase and Demonstration}

\author{John Doe}
\institute{Some Institution}
\date{\today}

\usepackage[english]{babel}
\usepackage{tikz}

\usetikzlibrary{decorations.markings}%
\usetikzlibrary{decorations}%
\usetikzlibrary{decorations}%
\usetikzlibrary{fadings}
\usetikzlibrary{patterns}

\begin{filecontents}{references.bib}
@book{Gould:1994aa,
	author = {Gould, Phillip L and Feng, Yuan},
	date-added = {2024-08-07 17:28:38 +0200},
	date-modified = {2024-08-07 17:28:49 +0200},
	keywords = {Solid Mechanics},
	publisher = {Springer},
	title = {Introduction to linear elasticity},
	volume = {2},
	year = {1994},
}

@book{Bathe:2006aa,
	author = {Bathe, Klaus-J{\"u}rgen},
	date-added = {2024-08-07 17:22:51 +0200},
	date-modified = {2024-08-07 17:23:00 +0200},
	keywords = {Finite Element Method},
	publisher = {Klaus-Jurgen Bathe},
	title = {Finite element procedures},
	year = {2006},
}

@book{Anand:2020aa,
	author = {Anand, Lallit and Govindjee, Sanjay},
	date-added = {2024-08-07 17:21:00 +0200},
	date-modified = {2024-08-07 17:22:21 +0200},
	keywords = {Solid Mechanics},
	publisher = {Oxford University Press},
	title = {Continuum mechanics of solids},
	year = {2020},
}
\end{filecontents}

\begin{document}

	\maketitle

	\begin{frame}[allowframebreaks]{Content}
		\tableofcontents
	\end{frame}

	\sectionframe{Demonstration of Individual Elements}
	\begin{frame}{Lists}
		\subsection{Enumerate}
		List using the '\texttt{enumerate}' environment:
		\begin{enumerate}
			\item First item.
			\item Second item.
			\begin{enumerate}
				\item Sub-list first item.
				\item Sub-list second item.
				\begin{enumerate}
					\item Sub-sub-list first item.
					\item Sub-sub-list second item.
				\end{enumerate}
			\end{enumerate}
		\end{enumerate}

		\vspace{5mm}
		\subsection{Itemize}
		List using the '\texttt{itemize}' environment:
		\begin{itemize}
			\item First item.
			\item Second item.
			\begin{itemize}
				\item Sub-list first item.
				\item Sub-list second item.
				\begin{itemize}
					\item Sub-sub-list first item.
					\item Sub-sub-list second item.
				\end{itemize}
			\end{itemize}
		\end{itemize}
	\end{frame}

	\begin{frame}{Alert \& Blocks}
		\subsection{Alert}
		\alert{Sample of the '\texttt{alert}' command.}

		\vspace{5mm}
		\subsection{Blocks}
		\begin{block}{Conventional Block}
			This block can be used to highlight key information of a given slide.
		\end{block}

		\begin{exampleblock}{Example Block}
			Examples of different concepts can be placed inside this block.
		\end{exampleblock}

		\begin{alertblock}{Alert Block}
			Furthermore, one can put very important information inside this block.
		\end{alertblock}
	\end{frame}

	\begin{frame}[fragile]{Code Blocks}
		\subsection{Code Blocks}

		\begin{lstlisting}[language = julia, caption = {Some random code}]
using Stela, Stela.Tensors

# Create some tensors
a = Tensors.rand(Float16, 5, 10)
b = Tensors.rand(Float16, 10, 3)

# Do some computations
c = sum(a * b)^Float16(4)

# Pass through the graph
forward(c)  # Compute c, result stored in c.data
backward(c) # Compute derivatives of c, stored in *.grad

println("dc/da = (a.grad)")
println("dc/db = (b.grad)")

# Visualize
to_dot_graph(c, "graph_file"; create_svg=true) # Export to Graphviz
		\end{lstlisting}

	\end{frame}

	\sectionframe{Practical Demonstration}
	\begin{frame}{Some Math \& Figures}

		\subsection{Math}
		\begin{columns}
			\column{0.6\textwidth}
				\begin{itemize}
					\item Equilibrium conditions
							\begin{subequations}
								\begin{align}
									-\nabla \cdot \boldsymbol{\sigma} &= \boldsymbol{p} &&\forall \boldsymbol{x} \in \Omega \\
									\boldsymbol{\sigma} \cdot \boldsymbol{n} &= \boldsymbol{t} &&\forall \boldsymbol{x} \in \Gamma_N \\
									\boldsymbol{u} &= \overline{\boldsymbol{u}} &&\forall \boldsymbol{x} \in \Gamma_D
								\end{align}
							\end{subequations}

					\item Linear strain-displacement relation
						\begin{equation}
							\boldsymbol{\varepsilon} = \frac{1}{2} \left( \nabla \boldsymbol{u} + \nabla \boldsymbol{u}^T \right)
						\end{equation}

					\item Constitutive equation
						\begin{equation}
							\boldsymbol{\sigma} = \boldsymbol{C} : \boldsymbol{\varepsilon}
						\end{equation}

					\item Resulting weighted-residual form
						\begin{equation}
							\begin{aligned}
								\int \limits_\Omega \boldsymbol{\varepsilon}(\boldsymbol{v}) : \boldsymbol{C} : \boldsymbol{\varepsilon}(\boldsymbol{u}_h) \,\textrm{d}\Omega
								&= \int \limits_\Omega \boldsymbol{v}\ \boldsymbol{p}\ \textrm{d}\Omega
								+ \int \limits_{\Gamma_N} \boldsymbol{v}\ \boldsymbol{t}\ \textrm{d}\partial\Omega
								+ \int \limits_{\Gamma_D} \boldsymbol{v}\ \boldsymbol{t}\ \textrm{d}\partial\Omega \\
								\land \quad \boldsymbol{u}_h &= \overline{\boldsymbol{u}} \quad \forall \boldsymbol{x} \in \Gamma_D \\
							\end{aligned}
						\end{equation}

				\end{itemize}

			\column{0.35\textwidth}

			\subsection{Figures}
			\begin{figure}
				\begin{center}
					\resizebox{!}{6.5cm}{
						\begin{tikzpicture}
							% b.c.
							\draw[default-fg, decorate, decoration = {
								markings,
								mark = between positions 0cm and 0.56cm step 0.8mm with {
									\draw[cyan] (0, 0) -- (0.1, -0.15);
								}
								pre = curveto,
								post = curveto,
								transform = {shift only},
							}]
							(1.5, 0.5) to[out = 10, in = -90]
							++(1.5, 2) to[out = 90, in = 0]
							++(-1, 1) to[out = 180, in = 30]
							++(-1, -1) to[out = -150, in = 90]
							++(-1, -1) to[out = -90, in = 190]
							cycle;

							\draw[default-fg, decorate, decoration = {
								markings,
								mark = between positions 4.5cm and 5cm step 1.2mm with {
									\draw[red, -stealth] (0, 0) -- (-0.2, 0.2);
								}
								pre = curveto,
								post = curveto,
								transform = {shift only},
							}]
							(1.5, 0.5) to[out = 10, in = -90]
							++(1.5, 2) to[out = 90, in = 0]
							++(-1, 1) to[out = 180, in = 30]
							++(-1, -1) to[out = -150, in = 90]
							++(-1, -1) to[out = -90, in = 190]
							cycle;

							% org. problem
							\draw[default-fg, fill = gray]
							(1.5, 0.5) to[out = 10, in = -90]
							++(1.5, 2) to[out = 90, in = 0]
							++(-1, 1) to[out = 180, in = 30]
							++(-1, -1) to[out = -150, in = 90]
							++(-1, -1) to[out = -90, in = 190]
							cycle;

							% shifted
							\begin{scope}[shift = {(0cm, 0.5cm)}, yscale = 1, rotate = -45]

								% b.c.
								\draw[default-fg, decorate, decoration = {
									markings,
									mark = between positions 3cm and 3.56cm step 0.8mm with {
										\draw[cyan] (0, 0) -- (0.1, -0.15);
									}
									pre = curveto,
									post = curveto,
									transform = {shift only},
								}]
								(1.5, -0.5) to[out = -10, in = 80]
								++(1, -2) to[out = -100, in = 0]
								++(-0.5, -1) to[out = 180, in = -30]
								++(-1.5, 1) to[out = 150, in = -90]
								++(-1, 0.5) to[out = 90, in = -190] (1.5, -0.5);

								\draw[default-fg, decorate, decoration = {
									markings,
									mark = between positions 6.6cm and 7.1cm step 1.2mm with {
										\draw[red, -stealth] (0, 0) -- (-0.2, 0.2);
									}
									pre = curveto,
									post = curveto,
									transform = {shift only},
								}]
								(1.5, -0.5) to[out = -10, in = 80]
								++(1, -2) to[out = -100, in = 0]
								++(-0.5, -1) to[out = 180, in = -30]
								++(-1.5, 1) to[out = 150, in = -90]
								++(-1, 0.5) to[out = 90, in = -190]
								(1.5, -0.5);

								\draw[default-fg, fill = gray]
								(1.5, -0.5) to[out = -10, in = 80]
								++(1, -2) to[out = -100, in = 0]
								++(-0.5, -1) to[out = 180, in = -30]
								++(-1.5, 1) to[out = 150, in = -90]
								++(-1, 0.5) to[out = 90, in = -190]
								cycle;

							\end{scope}

							\draw[default-fg, -latex] (2, -2) -- node[below, pos = 1] {$x_1$} ++(1, 0);
							\draw[default-fg, -latex] (2, -2) -- node[left, pos = 1] {$x_2$} ++(0, 1);

							% domains
							\node (omegaOrg) at (1.6, 1.6) {$\Omega$};
							\node (omegaDef) at (-0.4, -1.5) {$\hat{\Omega}$};

							\draw[default-fg, -latex] (omegaOrg) to[out = -145, in = 80 ] node[fill = default-bg] {$ + \boldsymbol{u}$} (omegaDef);

							% b.c.
							\node at (1.35, 3.65) {$\boldsymbol{t}$};
							\node at (1.05, 3.05) {$\Gamma_N$};

							\node at (1.9, 0.2) {$\overline{\boldsymbol{u}}$};
							\node at (2.6, 0.7) {$\Gamma_D$};

						\end{tikzpicture}
					}
					\caption{Initial $\Omega$ and deformed $\hat{\Omega}$ configuration of solid-mechanics problem.}
				\end{center}
			\end{figure}
		\end{columns}
	\end{frame}

	\section{Bibliography}
	\begin{frame}[allowframebreaks]{Literature}
		\nocite{*}
		\bibliographystyle{apalike}
		\bibliography{references.bib}
	\end{frame}

	\standoutframe{Thanks for looking at the theme!}{\weblink{github.com/schmaeke/BeamerDarkModern}}

\end{document}
